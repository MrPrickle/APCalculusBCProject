\documentclass{article}
\usepackage{graphicx} % Required for inserting images

\title{AP Calculus BC Project Topic 2 4th Hour}
\author{William Li, 10th Grade}
\date{May 2024}
\setlength{\parindent}{0pt}

\pdfpagewidth 8.5in
\pdfpageheight 11in

\topmargin 0in
\headheight 0in
\headsep 0in
\textheight 9in
\textwidth 6.5in
\oddsidemargin 0in
\evensidemargin 0in

% Macros
\newcommand \der {\frac {dy} {dx}}
\newcommand \ders {\left( \frac {dy} {dx} \right)^{2}}
\newcommand \dder {\dfrac {dy} {dx}}
\newcommand \derr {\frac {d^{2}y} {dx^2}}
\newcommand \derrr {\frac {d^{3}y} {dx^3}}
\newcommand \deralpha {\frac {d\beta} {d\alpha}}

\usepackage{amsmath} 
\usepackage[utf8]{inputenc}
\usepackage{comment}
\usepackage{ctex}

\usepackage{amssymb}


\def\quad{\hskip1em\relax}
\def\qquad{\hskip2em\relax}

\begin{comment}

\usepackage{ctex}
\usepackage{ctex}
\usepackage{ctex}

adding the following package, it resolves the error for \mathbb{R}
\usepackage{amssymb}

this section will be commenting out the whole block 
\addtolength{\voffset}{-4cm}
\addtolength{\voffset}{4cm}
\end{comment}

\begin{document}

\maketitle

\LARGE Part 1: Introduction

\hspace{1cm}

\Large Topic: Implicit Differentiation

Sources: AoPS Calculus Book, AoPS Calculus Lecture Problems, Demidovich Mathematical Analysis Book, Khan Academy, and Myself.



\newpage

\begin{comment}
	Implicit Differentiation Problems
\end{comment}

\LARGE Part 2: Problems

\vspace{2cm}

\large

\textbf{Problem 1} Find the slope of the tangent line to the circle $x^2 + y^2 = 1$ at the point $\left( \dfrac {\sqrt{2}} {2}, \dfrac {\sqrt {2}} {2} \right)$.\\
\textit{Source: Art of Problem Solving Calculus Book.}

\vspace{1cm}

\textbf{Problem 2} If $y^3 + 4y + 4 = x^2$, and both $x$ and $y$ are real numbers, what is the derivative of $y$ at $x = 2$?\\
\textit{Source: I made it up.}

\vspace{1cm}

\textbf{Problem 3} Find $\dfrac {dy} {dx}$ if $x^2 + y = \ln(y^2 - 1)$.\\
\textit{Source: Art of Problem Solving Calculus Book.}

\vspace{1cm}

\textbf{Problem 4} Find the first derivative of $\frac {y} {x} = \ln \sqrt{x^2 + y^2}$.\\
\textit{Source: Demidovich Mathematical Analysis Problems.}\\

\vspace{1cm}

\textbf{Problem 5} The graph of $xy^4 + x^2y = 10$ has a horizontal tangent at the point $(a , b)$. What is the value of $a^2b^4$?\\
\textit{Source: Art of Problem Solving College Math Forum}

\vspace{1cm}

\textbf{Problem 6} Prove the derivative of $y = a^x$ using implicit differentiation Hint: What if we take the $\ln$ of both sides?\\
\textit{Source: I made it up.}

\vspace{1cm}

\textbf{Problem 7} Find $y^{\prime} (2)$ if $y = 5x^{2 + 3x + 2xy}$, and $y(2) = -5$.\\
\textit{Source: Art of Problem Solving.}

\vspace{1cm}
\newpage

\textbf{Problem 8} Let $\beta$ be the dependent variable and let $f(\alpha, \beta) = \sin\alpha \cos\alpha + \sin\beta \cos\beta = 0.$ Prove that

\[d\alpha = -d\beta.\]

\textit{Source: I made it up.}

\vspace{1cm}

\textbf{Problem 9} There are 2 lines that are tangent to the ellipse $3x^2 - 72x + y^2 +26 = 0$, and pass through the origin. Find the equation of the line with the smaller slope.\\
\textit{Source: I made it up}

\vspace{1cm}

\textbf{Problem 10} 2 circles, have equations
\[y^2 + x^2 = 1.\]
and
\[y^2 + (x - 3)^2 = 4.\]
respectfully.

Find the equation of the line tangent to both circles.\\
\textit{Source: Art of Problem Solving.}

\vspace{1cm}

\textbf{Problem 11} Find $\dfrac {d^2y} {dx^2}$ if $y^2 = \dfrac {x^3} {4 - x}$.\\
\textit{Source: Khan Academy.}

\vspace{1cm}

\textbf{Problem 12} Find $\dfrac {d^2y} {dx^2}$ at $x = 1$ and $y = 2$, if $8x^2y + 2\ln(xy) = 7$.\\
\textit{Source: Khan Academy.}

\vspace{1cm}

\textbf{Problem 13} Find $\dfrac {d^3y} {dx^3}$ of $x^2 + 6xy + y^2 = 4$ on the point $\left( 1, 5 \right)$.\\
\textit{Source: I made it up.}

\vspace{1cm}

\textbf{Problem 14} Let $g\left( x, y \right) = 1 + xy + \left( xy \right)^2 + \left( xy \right)^3 + ... + \left( xy \right)^{100}$. If $\dfrac {dy} {dx} = -1$ for all real $x$ and $y$, find $g^{\prime}\left( \frac {1} {2}, 2 \right)$.\\
\textit{Source: I made it up.}

\vspace{1cm}

\begin{comment}
	\textbf{Problem 14} The 2 functions $4x^2 + ay^2 - 36 = 0$ and $bx^2 + 25y^2 -100 = 0$ have the same absolute minimum point. Find $ab$. \\ 
	\textit{Source: I made it up.}
	
	\vspace{1cm}
	
\end{comment}

\newpage

\textbf{Problem 15} Let $y = f(x)$ be a function of $x$ such that
\[x^2y^2 + x^2 + y^2 - 1 = 0.\]
Prove that:
\[\dfrac {dx} {\sqrt{1 - x^4}} + \dfrac {dy} {\sqrt{1 - y^4}} = 0.\]
\\
\textit{Source: Art of Problem Solving College Math Forum.}

\newpage

\begin{comment}
	Implicit Differentiation Solutions
\end{comment}

\LARGE Part 3: Solutions

\vspace{2cm}

\large

\textbf{Solution for Problem 1} First let's find the expression for the slope, which is $\dfrac {dy} {dx}$. 

\[2x + 2y\frac {dy} {dx} = 0\]
\[\frac {dy} {dx} = -\frac {x} {y}.\]

Thus, the slope at $\left( \dfrac {\sqrt{2}} {2}, \dfrac {\sqrt {2}} {2} \right)$ is $-\dfrac { \dfrac {\sqrt{2}} {2} } { \dfrac {\sqrt {2}} {2} } = -1.$\\

\textbf{Solution for Problem 2} First, we take the implicit derivative of our given function:\\

\[3y^2\der + 4\der = 2x\]

Moving the variables, we get

\[ \der \left( 3y^2 + 4 \right) = 2x \]
\[ \der = \frac {2x} {3y^{2} + 4}. \]

The problem asks us to find the derivative at $x = 2$, so by plugging in the $x$ value, we can find the $y$ value and calculate the derivative.

\[ y^3 + 4y + 4 = 2^2 \]
\[ y\left( y^2 + 4 \right) = 0 \]

We can see that $y = 0$ or $y^2 + 4 = 0$, but since $y$ is real, $y$ must be equal to $0$.

Thus, the derivative at $x = 2$ is $\dfrac {2 \cdot 2} {0 + 4} = \boxed{1}.$

\newpage

\textbf{Solution for Problem 3} Very straight-forward:\\

\[ 2x + \der = 2y \der \cdot \frac {1} {y^2 - 1} \]
\[ \der \left( 1 - \frac {2y} {y^2 - 1} \right) = -2x \]
\[ \der = \dfrac {2x} {\frac {2y - y^2 + 1} {y^2 - 1}} \]
\[ = \dfrac {2x\left( y^2 - 1 \right)} {-y^2 + 2y + 1} \]
\[ = \dfrac {2x - 2xy^{2}} {y^2 - 2y - 1} \]

\vspace{1cm}

\textbf{Solution for Problem 4} Start with quotient rule: \\

\[ \dfrac {x\der - y} {x^2} = \left( 2x + 2y\der \right) \cdot \frac {1} {2} \left( x^2 + y^2 \right)^{- \frac {1} {2}} \cdot \dfrac {1} {\sqrt {x^2 + y^2}} \]
\[ \dfrac {x\der - y} {x^2} = \frac {1} {2} \left( 2x + 2y\der \right) \cdot \dfrac {1} {x^2 + y^2} \]

\[ x\der - y = \dfrac {x^2 \cdot 2x} {2\left( x^2 + y^2 \right)} + \dfrac {x^2 \cdot 2y} {2 \left( x^2 + y^2 \right)} \der \]
\[ \left( x - \dfrac {x^2y} {x^2 + y^2} \right)\dder = \frac {x^3} {x^2 + y^2} + y \]
\[ \boxed{\dder = \dfrac {x^3 + x^2y + y^3} {x^3 + xy^2 - x^2y}}. \]


\vspace{1cm}

\newpage

\textbf{Solution for Problem 5} We start by finding $\der$ because that's the expression for the slope of a tangent line to the function. \\

\[ y^4 + 4xy^3\der + 2xy + \der x^2 = 0 \]
\[ \der \left( 4xy^3 + x^2 \right) = -y^4 - 2xy \]
\[ \der = \dfrac {-y^4 - 2xy} {4xy^3 + x^2} \]

We need $\der = 0$. Thus,

\[-y^4 - 2xy = 0\]
\[y^4 = 2xy\]
\[x = \frac {y^3} {2}\]

Plugging back into our original equation, we get:

\[ \frac {y^3} {2} \cdot y^4 + \frac {y^6} {4} \cdot y = 10 \]
\[ 3y^7 = 40 \]
\[ y = \sqrt[7]{\frac {40} {3}} \]

Now we can find the $x$ value

\[ x = \frac {1} {2} \left( \frac {40} {3} \right)^{\frac {3} {7}} \]

The problem asks for the $x$ value squared times the $y$ value to the power of 4.

\[ a^2b^4 = \frac {1} {4} \left( \frac {40} {3} \right)^{\frac {6} {7}} \cdot \left( \frac {40} {3} \right)^{\frac {4} {7}} \]
\[ = \frac {1} {4} \left( \frac {40} {3} \right)^{\frac {10} {7}} \]

\vspace{1cm}

\newpage

\textbf{Solution for Problem 6} First, we take the $\ln$ of both sides. \\

\[ \ln{y} = \ln{a^x} \]
\[ \ln{y} = x\ln{a} \]

Now we take the implicit differentiation:

\[ \der \frac {1} {y} = \ln{a} \]

Substitute $y = a^x$ back in recursively:

\[ \der \frac {1} {a^x} = \ln{a} \]
\[ \der = \ln{a}a^x. \]

\vspace{1cm}

\textbf{Solution for Problem 7} Very similar to the last problem, we start by taking the $\ln$ on both sides.\\

\[ \ln{y} =  \ln{5x^{2 + 3x + 2xy}} \]

Applying 2 log identities, we get: 

\[ \ln{y} = \ln{5} + \left( 2 + 3x + 2xy \right)\ln{x} \]

Now we take the implicit derivative

\[ \der \frac {1} {y} = \left( 3 + 2y + 2x\der \right)\ln{x} + \dfrac {2 + 3x + 2xy} {x} \]
\[ \der \left( \frac {1} {y} - 2x\ln{x} \right) = 3\ln{x} + 2y\ln{x} + \dfrac {2 + 3x + 2xy} {x} \]

Since the problem gave us $y(2) = -5$, they gave us the point $\left( 2, -5 \right)$ and we plug that in to get $y^{\prime}(2)$.
\[ y^{\prime}(2) = \dfrac {-7\ln{2} - 6} {-\frac {1} {5} - 4\ln{2}} \]
\[ = \boxed{\dfrac {35\ln{2} + 30} {20\ln{2} + 1}}. \]

\vspace{1cm}


\textbf{Solution for Problem 8} We start by taking the implicit derivative of the given function:\\

\[ -\cos\alpha \sin\alpha - \frac {d\beta} {d\alpha} \cos\beta \sin\beta = 0 \]
\[ \deralpha = \frac {\cos\alpha \sin\alpha} {\cos\beta \sin\beta}. \]

If we look back to the original function, we see that $\sin\alpha \cos\alpha = -\sin\beta \cos\beta$,

Thus,
\[ \deralpha = -1 \]
\[ d\beta = -d\alpha \]
\[ \boxed{d\alpha = -d\beta}. \]

\vspace{1cm}


\textbf{Solution for Problem 9} We start by finding the expression for the slope of a tangent line to the ellipse.\\

\[6x - 72 + 2y \der = 0\]
\[\der = \frac {72 - 6x} {2y} = \frac {36 - 3x} {y}.\]

For a line to pass through the origin and have the slope $\der$, it must be in the form:

\[y = \frac {36 - 3x} {y} x\]

Thus,

\[y^2 = 36x - 3x^2\]

Plugging back into the origin equation, we get:
\[3x^2 - 72x + 36x - 3x^2 + 26 = 0\]
\[-36x = -26\]
\[x = \frac {13} {18}.\]

Now we find $y$

\[y^2 = 36 \cdot \frac {13} {18} - 3 \left( \frac {13} {18} \right)^{2}\]
\[y^2 = 26 - \frac {507} {324}\]
\[y = \sqrt{\frac {7917} {324}}\]

Therefore, our equation of the line is:

\[y = \frac {36 - \frac {13} {18}} {\sqrt{\frac {7917} {324}}}x\]

\newpage


\textbf{Solution for Problem 10} First, we find the derivative expression for both circles.\\
Circle 1:
\[2y\der + 2x = 0\]
\[\der = -\frac {x} {y}.\]

Circle 2:
\[2y\der + 2x - 6 = 0\]
\[\der = \frac {3 - x} {y}.\]

Now we set them equal since the problem states that there is a line tangent to both circles.

\[-\frac {x} {y} \frac {3 - x} {y}.\]

Cross-multiplying, we get:

\[-xy = 3y - xy\]
\[3y = 0\]
\[y = 0.\]

Now we plug $y$ back into any of the two circles to find $x$.

\[0^2 + x^2 = 1\]
\[x = \pm 1.\]

Now how do we determine which one is the real $x$ value of their intersection? Since the circle equations are simple, we know that the second circle is located to the right of the first circle, which is centered at $(0, 0)$, so $x$ has to be $1$.

Thus,
\[\der = -\frac {1} {0} = \infty\]

This doesn't mean it doesn't exist, it means that the slope is perfectly vertical.

Thus, the equation of the tangent line is:

\[\boxed{x = 1}\]


\vspace{1cm}


\textbf{Solution for Problem 11} Get rid of fractions:\\

\[y^2\left( 4 - x \right) = x^3\]
\[4y^2 - xy^2 = x^3\]
\[8y\der - y^2 - 2xy\der = 3x^2\]
\[\der = \frac {3x^2 + y^2} {2xy - 8y}\]

Now we take the implicit derivative again to find the 2nd derivative:

\[8\der \der + 8y \derr - 2y\der - \left( 2y + 2x\der \right)\der - \derr 2xy = 0\]
\[\derr \left( 8y - 2xy\right) = 6x - 8\ders + 2y\der + 2y\der + 2x\ders \]
\[\derr = \dfrac {6x + \left( 2x - 8 \right) \ders + 4y\der} {8y - 2xy}\]
\[\derr = \dfrac {6x + \left( 2x - 8 \right)\left( \frac {3x^2 + y^2} {2xy - 8y} \right)^{2} + 4y\frac {3x^2 + y^2} {2xy - 8y}} {8y - 2xy}\]


\vspace{1cm}

\textbf{Solution for Problem 12} 

\[ 16xy + 8x^2\der + 2\left( y + x\der \right)\frac {1} {xy} = 0\]


\[16y + 16y\der + 16x\der + 8x^2\derr - 2x^{-2} - 2y^{-2}\der + 2y^{-1}\derr = 0\]

\[\der \left( \frac {2} {y} + 8x^2 \right) = -16xy - \frac {2} {x}\]
\[\der = \dfrac {-16xy - \frac {2} {x}} {8x^2 + \frac {2} {y}}\]

So our first derivative is:
\[= -\frac {8x^2y^2 + y} {4x^2y + x}\]

\newpage

Now let us find the second derivative:

\[\derr \left( 8x^2 + 2y^{-1} \right) = -16y - 16x\der = 16x\der + 2x^{-2} + 2y^{-2}\der\]
\[\derr = \dfrac {2x^{-2} - 16y + \left( 2y^{-2} - 32x \right)\der} {8x^2 + 2y^{-1}}\]
\[= \dfrac {\frac {2} {x} - 16y -\frac {8x^2y^2 + y} {4x^2y + x}\left( \frac {2} {y} - 32x \right)} {8x^2 + \frac {2} {y}}\]
\[= \dfrac {2 - 32 - \frac {34} {9} \cdot (-31)} {9}\]
\[ = \frac {-270 + 34 \cdot 31} {81} = \boxed{\frac {784} {81}}.\]

\vspace{1cm}

\textbf{Solution for Problem 13} Finding the first derivative: //

\[2x + 6y + 6x\der + 2y\der = 0\]
\[\der = \frac {-2x - 6y} {6x + 2y}\]
\[\der = -\frac {x + 3y} {3x + y}\]
\[\der = -2.\]

Now we find the second derivative:

\[2 + 6\der + 6\der + 6x\derr + 2\der \cdot \der + 2y \derr = 0\]

We will use the above expression to find the third derivative

\[\derr \left( 6x + 2y \right) = -2\ders - 12\der - 2\]
\[\derr = -\dfrac {2\ders + 12\der + 2} {6x + 2y}\]
\[\derr = -\frac {8 - 24 + 2} {6 + 10} = \frac {14} {16} = \frac {7} {8}.\]

Now let's find the third derivative:

\[6\derr + 6\derr + 6\derr + 6x\derrr + 2\derr \der + 2\der \derr + 2\der \derr + 2y \derrr = 0\]
\[\derrr \left( 6x + 2y \right) = -18\derr - 6\der \derr\]\[\derrr = -\dfrac {9\derr + 3\der \derr} {3x + 1}\]
\[\derrr = -\frac {9 \cdot \frac {7} {8} + 3 \cdot (-2) \cdot \frac {7} {8}} {3 + 1}\]
\[= -\frac {\frac {21} {8}} {4} = \boxed{-\frac {21} {32}}.\]

\vspace{1cm}


\textbf{Solution for Problem 14} This one is a really weird problem, but don't give up after just looking at it, give it a try, and find a pattern!\\

If we were to take the implicit derivative of the first 3 terms, we would get

\[0 + y + x\der + 2xy \left( y + x\der \right).\]

Since the problem gave us $\der = -1$, plugging it in would leave us with:

\[y - x + 2xy\left( y - x \right).\]

Notice that if we were to continue writing down terms, we would have a general structure of:

\[n\left(xy\right)^{n - 1}\left( y - x \right)\]

for each term. Also, the problem wanted us to find the derivative at $x = \frac {1} {2}$ and $y = 2$, which luckily for us, makes $xy = 1$. Using this, we get:

\[y - x + y - x + y - x + ...\]
\[= 2 - \frac {1} {2} + 2 - \frac {1} {2} + 2 - \frac {1} {2} + ...\]

With $99$ terms in total.

Thus, we have $2 \cdot 99 - \frac {1} {2} \cdot 99 = \boxed{\frac {297} {2}}.$

\vspace{1cm}


\textbf{Solution for Problem 15} Since nobody posted any solutions, the approach I thought of was pretty straightforward. So our goal is to prove that $\dfrac {dx} {\sqrt{1 - x^4}} + \dfrac {dy} {\sqrt{1 - y^4}} = 0$ is true. Notice that we have to get $1 - x^4$ and $1 - y^4$ somehow.

Let's start by finding $\dfrac {dy} {dx}$ of our given function, $f(x) = x^{2}y^{2} + x^{2} + y^{2} - 1 = 0.$

Using implicit differentiation, we get the following:

\[ 2xy^{2} + 2y\frac {dy} {dx}x^{2} + 2x + 2y\frac {dy} {dx} = 0 \]
\[ \frac {dy} {dx} \left( 2x^{2}y + 2y \right) = -2xy^{2} - 2x \]
\[ \dfrac {dy} {dx} = -\dfrac {2xy^{2} + 2x} {2x^{2}y + 2y}. \]

Factoring out $2x$ and $2y$ on the numerator and denominator respectfully, we get:

\[ \dfrac {dy} {dx} = -\dfrac {2x \left( y^{2} + 1 \right)} {2y \left( x^{2} + 1 \right)} \]

\[ \dfrac {dy} {dx} = -\dfrac {x \left( y^{2} + 1 \right)} {y \left( x^{2} + 1 \right)}. \]

We see that the numerator and denominator are nearly symmetrical with the $x$ swapped with $y$. So how do we get $1 - x^4$? If we find out how to get that, then we could do the same thing to get $1 - y^4$ because of symmetry.\\

This is where the hardest part of the problem comes into play.\\
First, from our original equation, notice that

\[x^{2}y^{2} + x^{2} + y^{2} - 1 = 0\]

can be turned into

\[x^{2} \left( y^2 + 1 \right) = 1 - y^2.\]

by factoring the $x^2$ and moving the rest to the right-hand side.\\

Thus, we can square both sides to get

\[ \left( \dfrac {dy} {dx} \right)^{2} = \dfrac {x^2 \left( y^{2} + 1 \right)^{2}} {y^2 \left( x^{2} + 1 \right)^{2}}. \]



Thus using symmetry (we could also factor $y^2$ out), our equation becomes:

\[ \left( \dfrac {dy} {dx} \right)^{2} = \dfrac {x^2 \left( y^{2} + 1 \right)^{2}} {y^2 \left( x^{2} + 1 \right)^{2}} = \dfrac {x^2 \left( y^{2} + 1 \right)\left( y^{2} + 1 \right)} {y^2 \left( x^{2} + 1 \right)\left( x^{2} + 1 \right)} = \dfrac {\left( 1 - y^2 \right) \left( y^2 + 1 \right)} {\left( 1 - x^2 \right) \left(x^2 + 1 \right)} \]

Because $\left( a^2 + b^2 \right) \left( a^2 - b^2 \right) = a^4 - b^4$ we get

\[ \left( \dfrac {dy} {dx} \right)^{2} = \dfrac {1 - y^4} {1 - x^4}. \]

Now we could take the square root of both sides.

\[\dfrac {dy} {dx} = -\dfrac {\sqrt{1 - y^4}} {\sqrt{1 - x^4}}.\]

Note that there has to be a negative there because when we squared both sides, there was also a negative there.

Cross-multiplying, we get

\[dy\sqrt{1 - x^4} = -dx\sqrt{1 - y^4}\]
\[dx\sqrt{1 - y^4} + dy\sqrt{1 - x^4} = 0\]

Finally, we divide both sides by $\sqrt{1 - x^4}\sqrt{1 - y^4}$ and get

\[ \boxed{\dfrac {dx} {\sqrt{1 - x^4}} + \dfrac {dy} {\sqrt{1 - y^4}} = 0}. \]

And we are done with our proof.






\end{document}
